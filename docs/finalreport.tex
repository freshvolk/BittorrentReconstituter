\documentclass{acm_proc_article-sp}

\usepackage{url}

\numberofauthors{3}
\author{
  \alignauthor
  Charlie Moore\\
  \email{cmoore@nmt.edu}
  \alignauthor
  Thomas Coppi\\
  \email{tcoppi@nmt.edu}
  \alignauthor
  Aaron Lovato\\
  \email{alovato@nmt.edu}
}
\title{BitTorrent Reconstitutor Project Final Report}

\begin{document}
\maketitle
\begin{abstract}
This paper describes the BitTorrent Reconstitutor project.
\end{abstract}
%\keywords{}

% Paper stuff starts here

\section{Introduction}
The BitTorrent Reconstitutor identifies BitTorrent file transfers in network
traffic and extracts the original file.  We assume that a full and complete
packet capture has been performed.  We will decode the traffic according to the
official BitTorrent
specification\footnote{\url{http://wiki.theory.org/BitTorrentSpecification}} and
ignore extensions for now.



\section{Problem Addressed}
Prior to this project, there was no publicly available tool for network
forensics analysts to use to extract files transferred over BitTorrent. Such
tools do exists for other protocols (such as tcpxtract
\footnote{\url{http://tcpxtract.sourceforge.net}}) so this tool extends the
capability to extract files from network traffic to include the BitTorrent
protocol. This gap in capabilities was deemed to be particularly interesting
because manually recovering a file transferred over BitTorrent from network
traffic would be impractical due to the variable number of peers and
piece-based distribution of files.


\section{Threat Model}
One of the ways of breaking up threats is by inbound and outbound threats.
Outbound threats consist mostly of data exfiltration problems.  Inbound attacks
deal with malware and contraband.  Both can also fall under the umbrella of
policy enforcement.

\subsection{Threats}
Threats this application can be used to mitigate include transfer of contraband
using BitTorrent, analysis of network traffic from an incident in which
BitTorrent was used to transfer files to a compromised host, and detection of
data exfiltration where BitTorrent is used to break the file(s) into smaller
pieces that may evade detection from file scanning utilities. Note that the
last case is one in which the application would be run against live traffic
while the first two would require offline processing of packet capture files.


\subsection{Use Cases}
The main motivation for this project is to give an analyst investigating an
incident involving BitTorrent a tool to identify the role BitTorrent played.  An
example of such a situation would be an investigation searching for child
pornography transferred across a network. Without a tool of this sort, analysts
would have to manually reconstruct the files and do so in such a way as to not
compromise their admissibility in court. Furthermore, the analyst would have to
ensure that all parts were received by the computer in question. We will attempt
to verify the successful download of all parts of the file in order to verify
the integrity of the data our tool produces.

Another situation in which this tool will be useful occurs when malware is
downloaded to a compromised computer using BitTorrent. If the attackers clean up
the compromised computer before a disk image is taken, network traffic
provides the only source of a sample of the malware used. Retrieving this file
or files from the network would be useful for further mitigation of the
attack. Also, identifying the peers involved in the download would allow action
to be taken to prevent them from being used in further attacks. To this end, our
tool will also list all IP addresses involved in the transfer of each file.

One last obvious use case is to mitigate exfiltration of data.  If the user is
using the BitTorrent protocol to send data out of the network, we would like to
know.  This can be accomplished by inspecting the files that we extract, and
possibly ensuring that the headers don't contain sensitive information, either.
This will require using a live capture stream.



\section{Approach}
We started this project by extensively reviewing the BitTorrent protocol
specification\footnote{\url{http://wiki.theory.org/BitTorrentSpecification}}.
Based on our own experiences as well as comments from our proposal presentation,
we found that many clients use unofficial extensions to the protocol, such as
using UDP instead of TCP. In light of the fact that reviewing and supporting
the many different implementations would require a much greater amount of time
than our project would allow, we decided to use the official specification to
decode traffic. Extending our project in the future to handle unofficial
implementations would entail replacing or modifying  a module based on the
specific extension desired. In order to make such modifications feasible, we
split our code into three modules: a packet handler, a BitTorrent session
identifier, and a file reconstructor.

\subsection{Packet Handler}
The packet handler performs the initial decoding of network packets up to the
transport layer. Since the official specification states that BitTorrent runs
over TCP, this module is built specifically to decode TCP/IP packets. It
produces a data structure that contains information needed to identify
BitTorrent sessions. Each of these data structures is written to a pipe for
concurrent processing by the next module. The main purpose of modularizing this
part of the code is to allow future extensions to support implementations of
the BitTorrent protocol that use a different transport layer protocol
(e.g., UDP instead of TCP). Building in a different module that produces the
same output would allow the rest of the code to operate as it is currently
designed without consideration of the underlying protocol.

\subsection{Session Finder}
The session finder module reads in the packet data structures from the packet
handler and parses them to identify BitTorrent sessions. A BitTorrent session is
defined by a tracker request for one or more files, the response, the pieces in
the actual file(s), and the tracker request stating that the download has
completed. When a "starting" request and corresponding response is found, the
session finder attempts to parse any non-tracker related packets into BitTorrent
pieces. If a packet is not a piece message, it will be discarded. Otherwise, the
session finder will either add a new piece to the corresponding session or add
the data to an unfinished piece. Once the "completed" tracker request is found,
the session is considered complete and is then written to a pipe for
processing. Incomplete sessions will not be processed because we cannot be sure
that the data in the session is meaningful until it is complete. The session
finder terminates when the packet handler closes the input pipe and all packets
have been processed.

\subsection{File Reconstructor}
XXX

\subsection{Tying It All Together}
We designed the system to run the modules concurrently, using pipes to
communicate with each other, for several reasons. The main motivation was to be
able to handle live input because sequential execution of the modules would not
be possible if the packet handler never exited. Since the packet handler would
never terminate when run with a live input, all of the modules need to be run in
parallel. Accordingly, the driver for the application sets up pipes and creates
3 processes to run each of the modules. Each module will block until its pipes
are opened by another module. The driver performs several other tasks including
parsing input options and cleanup of the pipes, but all of the major processing
is contained in the 3 modules it runs.

\subsection{Challenges Encountered}
The first challenge in this project came from the wide variety of unofficial
extensions to the  BitTorrent protocol used by various clients. As mentioned
above, our solution to this problem is to handle traffic according to the
official specification and design our code in a way that allows future
improvements to support new extensions to the protocol without redesigning the
whole project.

The next challenge we faced was discovered while designing the project.
Initially, we were planning to run each module successively to ensure all parts
of the input file had been processed by one module before the next module
started running. However, this design would have made working with a live input
nearly impossible and thus we worked on an alternate design, which led to the
concurrent processing used in our implementation.

\section{Deliverable}
\subsection{Functionality}
The application will identify and reconstruct any files transferred using the
BitTorrent protocol from network traffic. It can be run against both live input
and stored packet capture (pcap) files. It will take as input either a network
interface (for live input) or a pcap file and a list of \texttt{.torrent}
files. The \texttt{.torrent} files are necessary for verification of files found
in the network traffic; however, all files will be reconstructed, not just those
that can be verified. The application will produce the files as output along
with data pertaining to the transfer of the files as well as whether each file
was verified.

Specifically, this application will perform the following tasks: identify files,
reconstruct them, log information about the file transfer, and verify the
integrity of the file based on a \texttt{.torrent} file (if provided). It will not
perform any processing or analysis of the files other than that which is
necessary for identification and reconstruction of the files. Other applications
and/or manual analysis will be needed to process the files separately from this
application.


\section{Team Members}



\section{Forensics Impact}
This application will give analysts the ability to extract files transferred
using BitTorrent from network traffic. While tools (such as virus scanners) and
techniques (such as reverse engineering) are currently available and used to
analyze files, no publicly available application exists to reconstruct files
transferred over the BitTorrent protocol. This application is designed simply to
fill the gap in retrieving capabilities rather than extending any analysis
capability.

As noted in the threat model, most uses for this application will result in
processing of offline data. Accordingly, the main context for usage of this
application will be post-incident analysis, after packets captures have been
collected. However, some uses in which the application is processing live input
can be used to initially identify an incident rather than respond to a known
incident.

\subsection{Future}
As a possible future extension, DHT (distributed hash table) is close
to becoming standardized and is currently in use by many big BitTorrent names.
DHT makes it so that we don't have to use trackers and distributors to get the
peer and metadata information.  Torrent files can be distributed through the
DHT, which has a big impact on our program.  As of now, we're assuming that the
torrent file has to somehow magically be captured separately from the BitTorrent
traffic.  If we can incorporate DHT technology into our program, we can
eliminate this constraint.

\section{Conclusion}
In conclusion, this application solves the problem of being unable to extract
files encapsulated in the BitTorrent protocol from network traffic.  A company
or analyst would need this program for the use cases described above.  No
publicly available tool currently provides this functionality, so it fills a
hole in the network forensics analyst's toolkit.
%\balancecolumns

% Add References
%\bibliographystyle{plain}
%\bibliography{refs}

\end{document}
