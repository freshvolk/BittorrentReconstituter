\documentclass{beamer}

\usepackage{beamerthemesplit}
\setbeamertemplate{navigation symbols}{} % Get rid of stupid nav symbols
%\useoutertheme{infolines}
%\usetheme{Boadilla}
%\usetheme{Madrid}

\title{BitTorrent Reconstitutor Project}
%\subtitle{}
\author{Charlie Moore \and Aaron Lovato \and Thomas Coppi}
\date{December 8, 2009}
\institute[NMT]{New Mexico Tech}

\begin{document}

\section{Introduction}
\frame{\titlepage}
\frame{\tableofcontents}

\frame{
  \frametitle{Introduction}

   \begin{itemize}
   \item Identification and Extraction of files from network traffic
    \begin{itemize}
    \item Full packet capture necessary
    \end{itemize}
   \item Official specification~\cite{officialspec}
    \begin{itemize}
    \item Unofficial extensions not supported
    \end{itemize}
   \end{itemize}
}

\frame{
    \frametitle{Overview of the BitTorrent Protocol}

    \begin{itemize}
    \item Designed for efficient transfer of data over the Internet
    \item \texttt{.torrent} file used to describe each download
    \item Client-server model used for managing downloads
    \item Peer-to-peer download of data
    \begin{itemize}
        \item Many peers can be involved in a single download
        \item Having too many peers (more than 25) is unlikely to improve
        performance~\cite{btspec}
    \end{itemize}
    \item{Novel distribution of files}
    \begin{itemize}
        \item Files are broken down into \emph{pieces}
        \item Each peer can transmit individual pieces after they are received
    \end{itemize}
    \end{itemize}
}

\section{Problem Addressed}
\frame{
    \frametitle{What did we solve?}

    \begin{itemize}
    \item Difficult manual reconstruction of files
    \begin{itemize}
        \item Piecewise distribution of files
        \item Multiple peers
        \item Out of order pieces
    \end{itemize}
    \item No publicly available tool to perform this task automatically
    \begin{itemize}
        \item Tools exist for other file transfer protocols
    \end{itemize}
    \end{itemize}
}

\section{Threat Model}
\frame{
    \frametitle{Threat Model I}

    \begin{itemize}
        \item Categories of threats
        \item Incident investigation
        \begin{itemize}
            \item Inbound threat
            \item Network traffic may be the only way to recover files or data
            transferred to a compromised host
            \item BitTorrent could be used to make recovery of files extremely
            difficult
            \item The malware used is often of interest
        \end{itemize}
    \end{itemize}
}

\frame{
    \frametitle{Threat Model II}
    
    \begin{itemize}
        \item Exfiltration of sensitive data
        \begin{itemize}
            \item A host inside the network could be sending sensitive data out
            using BitTorrent to achieve rapid transfer of data and/or data
            obfuscation
            \item Also falls under incident investigation in certain
            circumstances
        \end{itemize}
        \item Transfer of contraband
        \begin{itemize}
            \item Difficult to categorize as either inbound or outbound
            \item Data gathered could be used in prosecution
        \end{itemize}
    \end{itemize}
}

\section{Approach}
\frame{
    \frametitle{Our Approach}

    \begin{itemize}
        \item Review of protocol specification
        \item An application that automatically reconstructs files transferred
        using BitTorrent from network traffic
        \item Ability to handle either saved packet captures or live
        input
        \item Must verify reconstructed files when possible
        \begin{itemize}
            \item Requires \texttt{.torrent} files for this task
        \end{itemize}
        \item Only supports official features of the protocol
        \begin{itemize}
            \item Can be extended in the future to handle other features
        \end{itemize}
    \end{itemize}
}

\section{Deliverable}
\frame{
    \frametitle{Design I}

    \begin{itemize}
        \item C++ with libpcap and Boost libraries
        \item Modules run in separate processes and communicate using pipes
        \item Runs multiple input files in parallel
        \item Packet Handler
        \begin{itemize}
            \item Abstracts out the underlying transport layer protocol
            \item Decodes relevant parts of raw packets and sends data for each
            packet to the Session Finder
        \end{itemize}
    \end{itemize}

}

\frame{
    \frametitle{Design II}

    \begin{itemize}
        \item Session Finder
        \begin{itemize}
            \item Decodes raw data into \emph{sessions}
            \item Starts a session with a tracker ``started'' request
            \item Gets peer information from tracker response(s)
            \item Only takes data from peers after a handshake is detected
            \item Decodes the \emph{pieces} transferred by each peer
            \item Ends session when tracker ``completed'' request is detected
        \end{itemize}
    \end{itemize}
}

\frame{
    \frametitle{Design III}

    \begin{itemize}
        \item File Reconstructor
    \end{itemize}
}
\section{Conclusion}
\frame{
%  \frametitle{Finally\dots}
}

\frame{
  \begin{center}
    \Large{Questions?}
  \end{center}
}

% Ifa we end up using references, put them in refs.bib and uncomment
\section{References}
\frame{
  \frametitle{References}
  \bibliographystyle{plain}
  \bibliography{finalrefs}
}

\end{document}
