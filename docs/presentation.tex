\documentclass{beamer}

\usepackage{beamerthemesplit}
\setbeamertemplate{navigation symbols}{} % Get rid of stupid nav symbols

\title{BitTorrent Reconstitutor Project}
\author{Charlie Moore \and Aaron Lovato \and Thomas Coppi}
\date{December 8, 2009}
\institute[NMT]{New Mexico Tech}

\begin{document}

\section{Introduction}
\frame{\titlepage}
\frame{\tableofcontents}

\frame{ \frametitle{Introduction}
  \begin{itemize}
  \item Identification and Extraction of files from network traffic
    \begin{itemize}
    \item Full packet capture necessary
    \item Torrent needs to be caught somehow to allow automatic multi-file mode
      extraction and verification
    \end{itemize}
  \item Official specification~\cite{officialspec}
    \begin{itemize}
    \item Unofficial extensions not supported
    \item DHT isn't supported
	\item Encryption isn't supported
    \end{itemize}
  \end{itemize}
}

\frame{ \frametitle{Overview of the BitTorrent Protocol}
  \begin{itemize}
  \item Designed for efficient transfer of data over the Internet
  \item \texttt{.torrent} file used to describe each download
  \item Client-server model used for managing downloads
  \item Peer-to-peer download of data --- Unicast transmission
    \begin{itemize}
    \item Many peers can be involved in a single download
    \item Having too many peers (more than 25) is unlikely to improve
      performance~\cite{btspec}
    \end{itemize}
  \item{Novel distribution of files}
    \begin{itemize}
    \item Files are broken down into \emph{pieces}
    \item Each peer can transmit individual pieces after they are received
    \end{itemize}
  \end{itemize}
}

\section{Problem Addressed}
\frame{ \frametitle{What forensics problem did we solve?}
  \begin{itemize}
  \item Difficult manual reconstruction of files
    \begin{itemize}
    \item Piecewise distribution of files
    \item Multiple peers
    \item Out of order pieces
    \end{itemize}
  \item No publicly available tool to perform this task automatically, but tools
    do exist for other protocols
  \end{itemize}
}

\section{Threat Model}
\frame[allowframebreaks]{ \frametitle{Threat Model}
  \begin{itemize}
  \item Categories of threats
  \item Incident investigation
    \begin{itemize}
    \item Inbound threat
    \item Network traffic may be the only way to recover files or data
      transferred to a compromised host
    \item BitTorrent could be used to make recovery of files extremely difficult
    \item The malware used is often of interest
    \end{itemize}

  \framebreak
  \item Exfiltration of sensitive data
    \begin{itemize}
    \item A host inside the network could be sending sensitive data out using
      BitTorrent to achieve rapid transfer of data and/or data obfuscation
    \item Also falls under incident investigation in certain circumstances
    \end{itemize}
  \item Transfer of contraband
    \begin{itemize}
    \item Difficult to categorize as either inbound or outbound
    \item Data gathered could be used in prosecution
    \end{itemize}
  \end{itemize}
}

\section{Approach}
\frame{ \frametitle{Approach \& Requirements}
  \begin{itemize}
  \item Review of protocol specification
  \item An application that automatically reconstructs files transferred using
    BitTorrent from network traffic
  \item Ability to handle either saved packet captures or live input
  \item Must verify reconstructed files when possible
    \begin{itemize}
    \item Requires \texttt{.torrent} files for this task
    \end{itemize}
  \item Only supports official features of the protocol
    \begin{itemize}
    \item Can be extended in the future to handle other features
    \end{itemize}
  \end{itemize}
}

\section{Deliverable}
\frame{ \frametitle{Design}
  \begin{itemize}
  \item C++ with libpcap and Boost libraries
  \item Modules run in separate processes and communicate using pipes
  \item Runs multiple input files in parallel
  \item Packet Handler
    \begin{itemize}
    \item Abstracts out the underlying transport layer protocol
    \item Decodes relevant parts of raw packets and sends data for each packet
      to the Session Finder
    \end{itemize}
  \end{itemize}
}

\frame{ \frametitle{Session Finder}
  \begin{itemize}
  \item Decodes raw data into \emph{sessions}
  \item Starts a session with a tracker ``started'' request
  \item Gets peer information from tracker response(s)
  \item Only takes data from peers after a handshake is detected
  \item Decodes the \emph{pieces} transferred by each peer
  \item Ends session when tracker ``completed'' or ``stopped'' request is detected
  \item Writes sessions to a pipe as soon as they are completed for processing
    by the File Reconstructor
  \end{itemize}
}

\frame{ \frametitle{File Reconstructor}
  \begin{itemize}
  \item Rebuilds the pieces of the file from a session
  \item Checks if a given \texttt{.torrent} file matches the session and if so
    splits and verifies the pieces
  \item If no \texttt{.torrent} file matches a session, a monolithic data file
    is created
    \begin{itemize}
    \item The monolithic data can be processed manually or with a file carving
      utility such as scalpel~\cite{scalpel}
    \end{itemize}
  \item Writes pieces in sequential order to output file named after the file's
    SHA-1 checksum
  \end{itemize}
}

\frame{ \frametitle{Final Product}
  \begin{itemize}
  \item Command-line application that can be run in either live or offline mode
    \begin{itemize}
    \item Live mode takes an interface name and reads traffic on that interface
    \item Offline mode takes one or more packet capture files and reads stored
      data
    \item Either mode can be given one or more \texttt{.torrent} files for use
      in verification and splitting of output files
    \end{itemize}
  \item Outputs information pertaining to each session as well as the downloaded
    file(s)
  \end{itemize}
}

\section{Conclusion}

\frame{
    \frametitle{Forensics Impact}
    \begin{itemize}
        \item First tool of its kind to reconstruct BitTorrent transfers
        \item Retrieval capability, not analysis
    \end{itemize}
}

\frame{ \frametitle{Forensics Impact}
  \begin{itemize}
  \item First tool of its kind to reconstruct BitTorrent transfers
  \item Retrieval capability, not analysis
  \end{itemize}
}

\frame{ \frametitle{Conclusion}
  \begin{itemize}
  \item Response to gap in capabilities
  \item Manual reconstruction is a difficult, time consuming task
  \item Developed an application to perform this task
    \begin{itemize}
    \item Multiple inputs for flexibility
    \item Performs verification of files when possible
    \item Can be extended in the future to support more features of the protocol
    \end{itemize}
  \item Helps save forensics analysts time and effort while improving
    reliability of data
  \item Can be applied to mitigate threats in real time
  \end{itemize}
}

\frame{
  \begin{center}
    \Large{Questions?}
  \end{center}
}

\frame[allowframebreaks]{ \frametitle{References}
  \bibliographystyle{plain}
  \bibliography{finalrefs}
}

\end{document}
% vim: tabstop=4:expandtab
