\documentclass[11pt,titlepage]{article}

% Give us palatino and optima
\usepackage[osf,sc]{mathpazo}
\usepackage{palatino}
\usepackage[T1]{fontenc}
\renewcommand{\sfdefault}{uop}

\usepackage{listings}
\usepackage{url}
\usepackage[letterpaper, bottom=1.5in, top=1.5in]{geometry}
\usepackage{fancyhdr}
\usepackage{microtype}

\pagestyle{fancy}
\lhead{CS 489} \chead{\today} \rhead{Design Document}
\cfoot{\thepage}

\author{Charlie Moore \and Aaron Lovato \and Thomas Coppi}
\title{CS 489 --- BitTorrent Reconstituter Design Document}

\begin{document}
\maketitle

\section{Introduction}
This document is still fairly high level for now.  We're fleshing it out as we
go.  Our project will consist of two modules.
% XXX Insert further discussion here.

\section{Pcap Parsing Module}
This first module will parse the \verb=pcap= file and give us BitTorrent
sessions.  It handles the protocol BitTorrent is running on, all other modules
are protocol agnostic with respect to the underlying protocol. This module will
require two concurrent processes in order to be able to handle live input. Once 
process will read packets in from libpcap, decode the TCP/IP headers, discard 
non-TCP packets, and build a Packet data structure to contain the relevant 
parts of the packet. These relevant parts of the packet are the 
source/destination ports and IP addresses and the TCP payload. This process 
will then serialize the data and send it over a pipe to process 2.

Process 2 will read data from a pipe and re-create the corresponding Packet data structures. It will then attempt to decode a tracker request. If this packet does contain a request, it will be recorded. If not, the process will then attempt to decode a tracker response. If a response is found, the process will make sure a corresponding request was already found. If not, the response will be discarded. Otherwise, relevant data will be pulled out of the response (peers, ports, etc.) and stored. 

If the packet is determined to contain neither a request or response, the 
process will attempt to decode the payload as a BitTorrent protocol packet. 
Depending on the type of packet, the data may be stored or discarded. We still 
need to determine how exactly the data from BitTorrent packets will be stored.
Non-BitTorrent traffic will be discarded.

\subsection{Data Structure}
% XXX Need some discussion of this data structure here
The Packet data structure has been completed. See Packet.hpp for details. One uncompleted data structure is a vector of structs containing the session info.
\begin{lstlisting}[language=C++]
struct Session:
  vector<Peer> peers // peers which participated in this session
  vector<PieceSHA1> checksums // must get from .torrent for later verification
  vector<string> bt_msgs // list of protocol messages all parsed and ready
\end{lstlisting}

\section{File Reconstruction Module}
This module will perform the work of rebuilding and verifying files based on a 
\texttt{.torrent} file and the data extracted from the network by the pcap 
parsing module. The following information is from our original design document, 
it may be useful for implementing the reconstruction module.

The next module will go through the sessions and build up a map for each file with
the pieces and other metadata that we need.

\subsection{Data Structure}
% XXX Need some discussion of this data structure here
\begin{lstlisting}[language=C++]
class File:
  vector<Map<PieceSHA1, PieceData> > pieces;
  // other crap
\end{lstlisting}

\end{document}
