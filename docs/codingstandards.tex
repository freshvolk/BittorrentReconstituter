\documentclass[11pt]{article}
\usepackage{palatino}
\usepackage[osf,sc]{mathpazo}

\usepackage{microtype}
\usepackage{url}
\usepackage[letterpaper, bottom=1.5in, top=1.5in]{geometry}

\usepackage{fancyhdr}
\pagestyle{fancy}
\lhead{CS 489} \chead{\today} \rhead{Coding Standards}
\cfoot{\thepage}

\author{Charlie Moore \and Aaron Lovato \and Thomas Coppi}
\title{CS 489 Bittorrent Project --- Coding Standards}

%===------------------------------------------------------------------------===%

\begin{document}

\section*{Coding Guidelines/Standards}
Our coding guidelines consist of the following rules:
\begin{itemize}
\item Files and Classes
\begin{itemize}
\item Each class should have a descriptive yet simple name formed of one or more words in mixed case, with the first letter of each word capitalized (e.g., \texttt{InputParser}).
\item Each class should be defined in a source file with the same name and a .cpp extension ((e.g., \texttt{InputParser.cpp}).
\item Likewise, each class should be declared in a header file with the same name and a .hpp extension (e.g., \texttt{InputParser.hpp}).
\end{itemize}
\item Method Naming and Scope
\begin{itemize}
\item Like classes, methods should also have a descriptive yet simple name formed of one or more words in mixed case. However, the first letter of the first word in a method name should not be capitalized while the first letter of each subsequent word should be capitalized (e.g., \texttt{readFile}).
\item Methods that are strictly of use within the class should be declared \texttt{private} while all other methods should be declared \texttt{public}.
\end{itemize}
\item Variable Naming and Scope
\begin{itemize}
\item Both class variables and local variables should follow the same naming convention as methods.
\item Unless absolutly necessary, class variables should be declared \texttt{private}.
\end{itemize}
\item Code Formatting
\begin{itemize}
\item Indentations will be \textbf{4 spaces} each.  Tabs will be converted to
  spaces.
\item The first curly brace will go on the same line as the statement it's
  linked to.
\item Each line (either code or comment)  should be kept at or under 80 characters in length, with newlines inserted as necessary to break down longer lines.

\item Algorithms and design decisions should be commented if they're not
  absolutely clear
\item The fundamental structure or layout of other people's code should not be
  modified without a group meeting where everyone agrees that the modification
  is an advantage. If a group member wishes to restructure existing code, this
  work MUST be done on a separate branch.
\item Branch-to-trunk merges will happen with group members that touched code on
  the branch \textit{physically present} so they can help resolve merge conflicts.
\end{itemize}
\end{itemize}
\end{document}
